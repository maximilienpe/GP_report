\documentclass{scrreprt}
\usepackage{array}
\usepackage{graphicx}
\usepackage{listings}
\usepackage{underscore}
\usepackage[bookmarks=true]{hyperref}
\usepackage[utf8]{inputenc}
\usepackage{float}
\usepackage[french]{babel}
\hypersetup{
    bookmarks=false,    % show bookmarks bar
    pdftitle={rapport_Lambolez_Petit},    % title
    pdfauthor={Théodore Lambolez, Maximilien Petit},                     % author
    pdfsubject={TeX and LaTeX},                        % subject of the document
    pdfkeywords={TeX, LaTeX, graphics, images}, % list of keywords
    colorlinks=true,       % false: boxed links; true: colored links
    linkcolor=blue,       % color of internal links
    citecolor=black,       % color of links to bibliography
    filecolor=black,        % color of file links
    urlcolor=black,        % color of external links
    linktoc=page            % only page is linked
}
\def\myversion{1.0}
\date{}
%\title
\usepackage{hyperref}
\begin{document}
\begin{figure}
   \begin{minipage}[c]{.46\linewidth}
      \includegraphics[scale=0.3]{images/telecom.png}
   \end{minipage} \hfill
   \begin{minipage}[c]{.46\linewidth}
      \includegraphics[scale=1.9]{images/lorraine.jpg}
   \end{minipage}
\end{figure}
\begin{flushright}
    \rule{15cm}{5pt}
    \vskip1cm
\end{flushright}
\begin{center}
	\vspace{3cm}
	\fbox{
	\begin{minipage}{0.9\textwidth}
        	\Huge{
			\textbf{
			\begin{center}
				Rapport \\Gestion de Production \\Lasurex
				\vspace{0.5cm}
			\end{center}
			}
		}
	\end{minipage}
	}
\end{center}
\begin{flushright}
        \vspace{5cm}
	\huge{
        \textbf{
	Ecrit par \\
	\vspace{0,875cm}
	\href{mailto:theodore.lambolez@telecomnancy.eu}{Théodore Lambolez} \\
	\href{mailto:maximilien.petit@telecomnancy.eu}{Maximilien Petit}\\
	}
	}
        \vspace{0,5cm}
        \large{
	\textbf{
	\today\\
	}	
	}
\end{flushright}

\tableofcontents

\chapter{Première partie}

\paragraph{Contextes et objectifs généraux}
Cet exercice s'inscrit dans le cours de Gestion de Production afin de mettre en pratique les notions
vues en cours sur un ERP pédagogique :  Prélude. 

Tout au long de cet exercice, nous nous intéresserons au cas de l'entreprise Lasurex qui fabrique deux
produits finis : les meubles lasurés (EML) et les meubles vernis (EMV). L'objectif de l'entreprise est 
au final de réusir à maximiser le taux de satisfaction client, de minimiser l'immobilisation des stocks et
le coût de réalisation de ces produits. 

Dans un premier temps, nous avons configuré l'ERP en le remplissant. Cette tâche nous a pris une séance de deux
heures de TP à l'AIP. En effet, il a fallut prendre en main le logiciel Prelude et comprendre dans quel ordre
remplir chacune des rubriques de l'onglet Données. 

\chapter{Deuxième partie} 


\paragraph{Notes}

On a fait un ordonnancement à capacité finie au plus tard, pour charger les machines au maximum. 
On a ajouté une machine de vernissage. On constate que le taux de charge demeure élevé, donc on peut 
tenter d'en remettre une autre dans ce même poste.

On a rajouté une deuxième machine vernissage. Le taux de charge ne change pas ! 

On a ajouté une équipe cette fois-ci. Toutes les OF ont été ordonnancées. Mais nous avons
trois OF en retard. Nous allons essayer de maximiser la satisfaction client. Un OF en retard.
Plus de stocks et plus de coût d'emploie. 


\end{document}
